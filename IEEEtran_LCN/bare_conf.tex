
%% bare_conf.tex
%% V1.3
%% 2007/01/11
%% by Michael Shell
%% See:
%% http://www.michaelshell.org/
%% for current contact information.
%%
%% This is a skeleton file demonstrating the use of IEEEtran.cls
%% (requires IEEEtran.cls version 1.7 or later) with an IEEE conference paper.
%%
%% Support sites:
%% http://www.michaelshell.org/tex/ieeetran/
%% http://www.ctan.org/tex-archive/macros/latex/contrib/IEEEtran/
%% and
%% http://www.ieee.org/

%%*************************************************************************
%% Legal Notice:
%% This code is offered as-is without any warranty either expressed or
%% implied; without even the implied warranty of MERCHANTABILITY or
%% FITNESS FOR A PARTICULAR PURPOSE! 
%% User assumes all risk.
%% In no event shall IEEE or any contributor to this code be liable for
%% any damages or losses, including, but not limited to, incidental,
%% consequential, or any other damages, resulting from the use or misuse
%% of any information contained here.
%%
%% All comments are the opinions of their respective authors and are not
%% necessarily endorsed by the IEEE.
%%
%% This work is distributed under the LaTeX Project Public License (LPPL)
%% ( http://www.latex-project.org/ ) version 1.3, and may be freely used,
%% distributed and modified. A copy of the LPPL, version 1.3, is included
%% in the base LaTeX documentation of all distributions of LaTeX released
%% 2003/12/01 or later.
%% Retain all contribution notices and credits.
%% ** Modified files should be clearly indicated as such, including  **
%% ** renaming them and changing author support contact information. **
%%
%% File list of work: IEEEtran.cls, IEEEtran_HOWTO.pdf, bare_adv.tex,
%%                    bare_conf.tex, bare_jrnl.tex, bare_jrnl_compsoc.tex
%%*************************************************************************

% *** Authors should verify (and, if needed, correct) their LaTeX system  ***
% *** with the testflow diagnostic prior to trusting their LaTeX platform ***
% *** with production work. IEEE's font choices can trigger bugs that do  ***
% *** not appear when using other class files.                            ***
% The testflow support page is at:
% http://www.michaelshell.org/tex/testflow/



% Note that the a4paper option is mainly intended so that authors in
% countries using A4 can easily print to A4 and see how their papers will
% look in print - the typesetting of the document will not typically be
% affected with changes in paper size (but the bottom and side margins will).
% Use the testflow package mentioned above to verify correct handling of
% both paper sizes by the user's LaTeX system.
%
% Also note that the "draftcls" or "draftclsnofoot", not "draft", option
% should be used if it is desired that the figures are to be displayed in
% draft mode.
%
\documentclass[conference]{IEEEtran}
% Add the compsoc option for Computer Society conferences.
%
% If IEEEtran.cls has not been installed into the LaTeX system files,
% manually specify the path to it like:
% \documentclass[conference]{../sty/IEEEtran}





% Some very useful LaTeX packages include:
% (uncomment the ones you want to load)


% *** MISC UTILITY PACKAGES ***
%
\usepackage{color}
\usepackage{ifpdf}
\usepackage{filecontents}
% Heiko Oberdiek's ifpdf.sty is very useful if you need conditional
% compilation based on whether the output is pdf or dvi.
% usage:
 \ifpdf
   \usepackage[pdftex]{graphicx}
   \DeclareGraphicsExtensions{.pdf,.PNG,.jpg,.jpeg,.mps, .PNG}
   \usepackage{pgf}
   \usepackage{tikz}
 \else
   \usepackage{graphicx}
   \DeclareGraphicsExtensions{.eps,.bmp}
   \DeclareGraphicsRule{.emf}{bmp}{}{}% declare EMF filename extension
   \DeclareGraphicsRule{.PNG}{bmp}{}{}% declare PNG filename extension
   \usepackage{pgf}
   \usepackage{tikz}
   \usepackage{pstricks}
   % dvi code
 \fi
 \usepackage{epic,bez123}
 \usepackage{floatflt}% package for floatingfigure environment
 \usepackage{wrapfig}% package for wrapfigure environment
% The latest version of ifpdf.sty can be obtained from:
% http://www.ctan.org/tex-archive/macros/latex/contrib/oberdiek/
% Also, note that IEEEtran.cls V1.7 and later provides a builtin
% \ifCLASSINFOpdf conditional that works the same way.
% When switching from latex to pdflatex and vice-versa, the compiler may
% have to be run twice to clear warning/error messages.
\usepackage{verbatimbox}

\usepackage{array}
\newcolumntype{L}[1]{>{\raggedright\let\newline\\\arraybackslash\hspace{0pt}}m{#1}}
\newcolumntype{C}[1]{>{\centering\let\newline\\\arraybackslash\hspace{0pt}}m{#1}}
\newcolumntype{R}[1]{>{\raggedleft\let\newline\\\arraybackslash\hspace{0pt}}m{#1}}



% *** CITATION PACKAGES ***
%
%\usepackage{cite}
% cite.sty was written by Donald Arseneau
% V1.6 and later of IEEEtran pre-defines the format of the cite.sty package
% \cite{} output to follow that of IEEE. Loading the cite package will
% result in citation numbers being automatically sorted and properly
% "compressed/ranged". e.g., [1], [9], [2], [7], [5], [6] without using
% cite.sty will become [1], [2], [5]--[7], [9] using cite.sty. cite.sty's
% \cite will automatically add leading space, if needed. Use cite.sty's
% noadjust option (cite.sty V3.8 and later) if you want to turn this off.
% cite.sty is already installed on most LaTeX systems. Be sure and use
% version 4.0 (2003-05-27) and later if using hyperref.sty. cite.sty does
% not currently provide for hyperlinked citations.
% The latest version can be obtained at:
% http://www.ctan.org/tex-archive/macros/latex/contrib/cite/
% The documentation is contained in the cite.sty file itself.






% *** GRAPHICS RELATED PACKAGES ***
%
\ifCLASSINFOpdf
  % \usepackage[pdftex]{graphicx}
  % declare the path(s) where your graphic files are
  % \graphicspath{{../pdf/}{../jpeg/}}
  % and their extensions so you won't have to specify these with
  % every instance of \includegraphics
  % \DeclareGraphicsExtensions{.pdf,.jpeg,.png}
\else
  % or other class option (dvipsone, dvipdf, if not using dvips). graphicx
  % will default to the driver specified in the system graphics.cfg if no
  % driver is specified.
  % \usepackage[dvips]{graphicx}
  % declare the path(s) where your graphic files are
  % \graphicspath{{../eps/}}
  % and their extensions so you won't have to specify these with
  % every instance of \includegraphics
  % \DeclareGraphicsExtensions{.eps}
\fi
% graphicx was written by David Carlisle and Sebastian Rahtz. It is
% required if you want graphics, photos, etc. graphicx.sty is already
% installed on most LaTeX systems. The latest version and documentation can
% be obtained at: 
% http://www.ctan.org/tex-archive/macros/latex/required/graphics/
% Another good source of documentation is "Using Imported Graphics in
% LaTeX2e" by Keith Reckdahl which can be found as epslatex.ps or
% epslatex.pdf at: http://www.ctan.org/tex-archive/info/
%
% latex, and pdflatex in dvi mode, support graphics in encapsulated
% postscript (.eps) format. pdflatex in pdf mode supports graphics
% in .pdf, .jpeg, .png and .mps (metapost) formats. Users should ensure
% that all non-photo figures use a vector format (.eps, .pdf, .mps) and
% not a bitmapped formats (.jpeg, .png). IEEE frowns on bitmapped formats
% which can result in "jaggedy"/blurry rendering of lines and letters as
% well as large increases in file sizes.
%
% You can find documentation about the pdfTeX application at:
% http://www.tug.org/applications/pdftex





% *** MATH PACKAGES ***
%
%\usepackage[cmex10]{amsmath}
% A popular package from the American Mathematical Society that provides
% many useful and powerful commands for dealing with mathematics. If using
% it, be sure to load this package with the cmex10 option to ensure that
% only type 1 fonts will utilized at all point sizes. Without this option,
% it is possible that some math symbols, particularly those within
% footnotes, will be rendered in bitmap form which will result in a
% document that can not be IEEE Xplore compliant!
%
% Also, note that the amsmath package sets \interdisplaylinepenalty to 10000
% thus preventing page breaks from occurring within multiline equations. Use:
%\interdisplaylinepenalty=2500
% after loading amsmath to restore such page breaks as IEEEtran.cls normally
% does. amsmath.sty is already installed on most LaTeX systems. The latest
% version and documentation can be obtained at:
% http://www.ctan.org/tex-archive/macros/latex/required/amslatex/math/





% *** SPECIALIZED LIST PACKAGES ***
%
%\usepackage{algorithmic}
% algorithmic.sty was written by Peter Williams and Rogerio Brito.
% This package provides an algorithmic environment fo describing algorithms.
% You can use the algorithmic environment in-text or within a figure
% environment to provide for a floating algorithm. Do NOT use the algorithm
% floating environment provided by algorithm.sty (by the same authors) or
% algorithm2e.sty (by Christophe Fiorio) as IEEE does not use dedicated
% algorithm float types and packages that provide these will not provide
% correct IEEE style captions. The latest version and documentation of
% algorithmic.sty can be obtained at:
% http://www.ctan.org/tex-archive/macros/latex/contrib/algorithms/
% There is also a support site http://matplotlib.org/api/colors_api.htmlat:
% http://algorithms.berlios.de/index.html
% Also of interest may be the (relatively newer and more customizable)
% algorithmicx.sty package by Szasz Janos:
% http://www.ctan.org/tex-archive/macros/latex/contrib/algorithmicx/




% *** ALIGNMENT PACKAGES ***
%
%\usepackage{array}
% Frank Mittelbach's and David Carlisle's array.sty patches and improves
% the standard LaTeX2e array and tabular environments to provide better
% appearance and additional user controls. As the default LaTeX2e table
% generation code is lacking to the point of almost being broken with
% respect to the quality of the end results, all users are strongly
% advised to use an enhanced (at the very least that provided by array.sty)
% set of table tools. array.sty is already installed on most systems. The
% latest version and documentation can be obtained at:
% http://www.ctan.org/tex-archive/macros/latex/required/tools/


%\usepackage{mdwmath}
%\usepackage{mdwtab}
% Also highly recommended is Mark Wooding's extremely powerful MDW tools,
% especially mdwmath.sty and mdwtab.sty which are used to format equations
% and tables, respectively. The MDWtools set is already installed on most
% LaTeX systems. The lastest version and documentation is available at:
% http://www.ctan.org/tex-archive/macros/latex/contrib/mdwtools/


% IEEEtran contains the IEEEeqnarray family of commands that can be used to
% generate multiline equations as well as matrices, tables, etc., of high
% quality.


%\usepackage{eqparbox}
% Also of notable interest is Scott Pakin's eqparbox package for creating
% (automatically sized) equal width boxes - aka "natural width parboxes".
% Available at:
% http://www.ctan.org/tex-archive/macros/latex/contrib/eqparbox/





% *** SUBFIGURE PACKAGES ***
%\usepackage[tight,footnotesize]{subfigure}
% subfigure.sty was written by Steven Douglas Cochran. This package makes it
% easy to put subfigures in your figures. e.g., "Figure 1a and 1b". For IEEE
% work, it is a good idea to load it with the tight package option to reduce
% the amount of white space around the subfigures. subfigure.sty is already
% installed on most LaTeX systems. The latest version and documentation can
% be obtained at:
% http://www.ctan.org/tex-archive/obsolete/macros/latex/contrib/subfigure/
% subfigure.sty has been superceeded by subfig.sty.



%\usepackage[caption=false]{caption}
%\usepackage[font=footnotesize]{subfig}
% subfig.sty, also written by Steven Douglas Cochran, is the modern
% replacement for subfigure.sty. However, subfig.sty requires and
% automatically loads Axel Sommerfeldt's caption.sty which will override
% IEEEtran.cls handling of captions and this will result in nonIEEE style
% figure/table captions. To prevent this problem, be sure and preload
% caption.sty with its "caption=false" package option. This is will preserve
% IEEEtran.cls handing of captions. Version 1.3 (2005/06/28) and later 
% (recommended due to many improvements over 1.2) of subfig.sty supports
% the caption=false option directly:
%\usepackage[caption=false,font=footnotesize]{subfig}
%
% The latest version and documentation can be obtained at:
% http://www.ctan.org/tex-archive/macros/latex/contrib/subfig/
% The latest version and documentation of caption.sty can be obtained at:
% http://www.ctan.org/tex-archive/macros/latex/contrib/caption/


\usepackage{caption}
\usepackage{tikz}
\def\checkmark{\tikz\fill[scale=0.4](0,.35) -- (.25,0) -- (1,.7) -- (.25,.15) -- cycle;}

% *** FLOAT PACKAGES ***
%
%\usepackage{fixltx2e}
% fixltx2e, the successor to the earlier fix2col.sty, was written by
% Frank Mittelbach and David Carlisle. This package corrects a few problems
% in the LaTeX2e kernel, the most notable of which is that in current
% LaTeX2e releases, the ordering of single and double column floats is not
% guaranteed to be preserved. Thus, an unpatched LaTeX2e can allow a
% single column figure to be placed prior to an earlier double column
% figure. The latest version and documentation can be found at:
% http://www.ctan.org/tex-archive/macros/latex/base/



%\usepackage{stfloats}
% stfloats.sty was written by Sigitas Tolusis. This package gives LaTeX2e
% the ability to do double column floats at the bottom of the page as well
% as the top. (e.g., "\begin{figure*}[!b]" is not normally possible in
% LaTeX2e). It also provides a command:
%\fnbelowfloat
% to enable the placement of footnotes below bottom floats (the standard
% LaTeX2e kernel puts them above bottom floats). This is an invasive package
% which rewrites many portions of the LaTeX2e float routines. It may not work
% with other packages that modify the LaTeX2e float routines. The latest
% version and documentation can be obtained at:
% http://www.ctan.org/tex-archive/macros/latex/contrib/sttools/
% Documentation is contained in the stfloats.sty comments as well as in the
% presfull.pdf file. Do not use the stfloats baselinefloat ability as IEEE
% does not allow \baselineskip to stretch. Authors submitting work to the
% IEEE should note that IEEE rarely uses double column equations and
% that authors should try to avoid such use. Do not be tempted to use the
% cuted.sty or midfloat.sty packages (also by Sigitas Tolusis) as IEEE does
% not format its papers in such ways.





% *** PDF, URL AND HYPERLINK PACKAGES ***
%
%\usepackage{url}
% url.sty was written by Donald Arseneau. It provides better support for
% handling and breaking URLs. url.sty is already installed on most LaTeX
% systems. The latest version can be obtained at:
% http://www.ctan.org/tex-archive/macros/latex/contrib/misc/
% Read the url.sty source comments for usage information. Basically,
% \url{my_url_here}.





% *** Do not adjust lengths that control margins, column widths, etc. ***
% *** Do not use packages that alter fonts (such as pslatex).         ***
% There should be no need to do such things with IEEEtran.cls V1.6 and later.
% (Unless specifically asked to do so by the journal or conference you plan
% to submit to, of course. )


% correct bad hyphenation here
\hyphenation{op-tical net-works semi-conduc-tor}

\begin{document}
%
% paper title
% can use linebreaks \\ within to get better formatting as desired
\title{Incorporating Multiple Cluster Models for Network Traffic Classification}


% author names and affiliations
% use a multiple column layout for up to three different
% affiliations
\author{\IEEEauthorblockN{Anil Kumar, Jinoh Kim, Sang C. Suh}
	\IEEEauthorblockA
	{Department of Computer Science \\
		Texas A\&M University,
		Commerce, TX\\
		Email:akatta1@leomail.tamuc.edu, \\
		jinoh.kim@tamuc.edu, sang.suh@tamuc.edu}
	\and
	\IEEEauthorblockN{Ganho Choi}
	\IEEEauthorblockA{Sysmate Inc.\\
		1290 Dunsan-Dong Seo-Gu, Deajeon, 302-830, Korea\\
		Email: ghchoi@sysmate.com}
}
% conference papers do not typically use \thanks and this command
% is locked out in conference mode. If really needed, such as for
% the acknowledgment of grants, issue a \IEEEoverridecommandlockouts
% after \documentclass

% for over three affiliations, or if they all won't fit within the width
% of the page, use this alternative format:
% 
%\author{\IEEEauthorblockN{Michael Shell\IEEEauthorrefmark{1},
%Homer Simpson\IEEEauthorrefmark{2},
%James Kirk\IEEEauthorrefmark{3}, 
%Montgomery Scott\IEEEauthorrefmark{3} and
%Eldon Tyrell\IEEEauthorrefmark{4}}
%\IEEEauthorblockA{\IEEEauthorrefmark{1}School of Electrical and Computer Engineering\\
%Georgia Institute of Technology,
%Atlanta, Georgia 30332--0250\\ Email: see http://www.michaelshell.org/contact.html}
%\IEEEauthorblockA{\IEEEauthorrefmark{2}Twentieth Century Fox, Springfield, USA\\
%Email: homer@thesimpsons.com}
%\IEEEauthorblockA{\IEEEauthorrefmark{3}Starfleet Academy, San Francisco, California 96678-2391\\
%Telephone: (800) 555--1212, Fax: (888) 555--1212}
%\IEEEauthorblockA{\IEEEauthorrefmark{4}Tyrell Inc., 123 Replicant Street, Los Angeles, California 90210--4321}}




% use for special paper notices
%\IEEEspecialpapernotice{(Invited Paper)}




% make the title area
\maketitle

\begin{abstract}
	%Identifying applications are critical for a broad range of network related activities like bandwidth usage, security etc. Earlier, applications are identified based on port numbers, which proved to be not accurate anymore; based of payload signatures, which is proved to be accurate but has been limited in the real world implementation because of privacy concerns; based on flow statistics, which uses machine learning algorithms to find the patterns in the flow statistics and use it in classification, which has been widely used for many classification problems.  In this research, we explore the importance of the attributes or a combination of flow attributes which can classify applications effectively. The idea is to combine clustering and using combinations of flow attributes and we measure accuracy of each combination. We are currently evaluating our model with real-world traffic traces indicating effectiveness of the selective attributes is effective than using the whole set of attributes.
	Network traffic classification is one of essential functions for local and ISP networks for quality of service, network usage statistics, resource provisioning, and security. With its importance, a substantial number of previous studies have explored various machine learning techniques based on network flow statistics to improve the accuracy of classification and reported promising results with fairly high classification accuracy. However, what we observed from previously proposed network traffic classification techniques with our own data set recently collected are somewhat unacceptable results. In particular, we observed that simply combining flow attributes for classification may lead to unexpectedly poor accuracy in classification (less than xx\%).
	In this paper, we propose a new traffic classification method based on attribute groups, each of which consists of a set of attributes belonging to a single parameter (e.g., packet size, inter-arrival time, etc).
	Our method then incorporates multiple cluster models obtained from individual attribute groups to reach the final classification decision based on the population of candidate protocols (or applications).
	From our extensive experiments, we observed that our proposed technique significantly outperforms existing cluster-based classification techniques, showing up to yy\% better accuracy.
\end{abstract}
% IEEEtran.cls defaults to using nonbold math in the Abstract.
% This preserves the distinction between vectors and scalars. However,
% if the conference you are submitting to favors bold math in the abstract,
% then you can use LaTeX's standard command \boldmath at the very start
% of the abstract to achieve this. Many IEEE journals/conferences frown on
% math in the abstract anyway.

% no keywords




% For peer review papers, you can put extra information on the cover
% page as needed:
% \ifCLASSOPTIONpeerreview
% \begin{center} \bfseries EDICS Category: 3-BBND \end{center}
% \fi
%
% For peerreview papers, this IEEEtran command inserts a page break and
% creates the second title. It will be ignored for other modes.
% \IEEEpeerreviewmaketitle



\section{Introduction}

Accurately identifying network-based applications is of major interest for local and ISP networks for various purposes, including
quality of service, network usage statistics, resource provisioning, and security~\cite{Bernaille:2006:EAI:1368436.1368445, DBLP:conf/iwcmc/GrimaudoMB12, DBLP:conf/infocom/XieIKFN12}.
Earlier, network protocols/applications were identified simply based on TCP/UDP port numbers.
%, but the traditional identification technique is no longer reliable since many of existing and newly emerging network applications allow dynamic port selection without relying on standard port numbers~\cite{Ohzahata:2005, BitTorrentProto}.
However, the traditional technique based on port-numbers are proved to be ineffective where accuracy is less than 70\%~\cite{ACAS} since network applications using random port numbers or non-standard port numbers are increasing day-by-day and also usage of tunneling makes identification of applications more difficult just based on port numbers.
Due to this reason, a substantial body of research has been conducted to replace or complement the port-based identification. 

One approach to overcome the limitation of the port-based identification is to inspect the packet payload information with template signature sets~\cite{conf/IEEEcit/YeXWP09,DBLP:conf/noms/ParkWKH08} or machine learning techniques~\cite{ACAS}.
While highly accurate,  drawbacks of the deep packet inspection-based approach include encrypted traffic transformed with cryptographic keys and privacy concerns as many countries do not permit the extraction of full payload information from packets with increasing privacy requirements.

The limitations of the traditional port-based identification and the payload inspection-based classification suggested to utilize  transport layer characteristics of the application as the differentiator. From previously proposed techniques~\cite{}, we can see the combination of transport layer characteristics with machine learning techniques would be an effective alternative for network traffic classification. %accurately identifies the applications, with accuracies comparable to that of payload-based signature matching.
Several techniques were proposed based on supervised learning~\cite{}, while some other techniques utilized unsupervised or semi-supervised clustering techniques~\cite{}, reporting promising  accuracy for network traffic classification.
However, what we actually observed from previously proposed network traffic classification techniques with our own data set recently collected are somewhat unacceptable results. In particular, one important observation is that simply combining flow attributes for classification leads to unexpectedly poor accuracy in classification (less than xx\%), which motivates us to thoroughly examine the impact of flow attributes in this work.\footnote{A (network) flow is defined as a set of packets for a single session of communication with the five tuples of source IP address, destination IP address, source port number, destination port number, and protocol type.} 

%An alternative approach is to utilize machine learning-based classification techniques with flow statistics, such as flow size, flow duration, packet size, packet inter-arrival time, and so forth, without having privacy concerns~\cite{}.

To evaluate the significance of flow attributes to classification accuracy, we use a notion of attribute groups.
%In this paper, we propose a new traffic classification method based on clustering with the
An attribute group consists of a set of attributes that are derived from a single communication characteristic.
For example, the packet size group includes the minimum packet size, maximum packet size, average packet size, standard deviation of packet sizes observed from a single flow.
We consider four attribute groups of flow information group, packet size group, packet inter-arrival time group, and relative packet inter-arrival time group, as will be discussed in the Section~\ref{sec:grouping} in detail.
From our preliminary experiments, we observed that some combinations of attributes work quite better than the other combinations.
Moreover, simply applying a subset of attributes selected based on the evaluation to supervised techniques significantly improves performance compared to using the entire attributes without selection.
With the initial observations, we developed a new semi-supervised learning technique based on the attribute groups.
A key challenge for this approach is how to use multiple attribute groups to make a single classification decision.
To address this, we establish independent cluster models based on individual attribute groups and incorporate the results collected from multiple cluster models. 
Although it is known that clustering techniques generally work poorly compared to supervised learning techniques~\cite{DBLP:conf/infocom/XieIKFN12}, we will present that the proposed technique using multiple cluster models yield comparable classification accuracy.

The key contributions of this paper can be summarized as follows:

\begin{itemize}
	\item We evaluate the significance of flow attributes to classification accuracy with a notion of attribute group.
	We used 18 flow attributes in total and four groups are formed, which are flow information group, packet size group, packet inter-arrival time group, and relative packet inter-arrival time group.
	\item From the evaluation results with the attribute groups, we examine classification accuracy with the entire attributes and with the selected ones using supervised learning methods, to ensure  validity of the selection. 
	\item We present a new clustering technique for network traffic classification that utilizes multiple cluster models developed from the attribute groups. A set of heuristic algorithms are also presented to incorporate multiple cluster models.
	\item We also present experimental results for evaluating the proposed traffic classification technique. Experiments for sensitivity study are also conducted to see the impact of configurable parameters.
\end{itemize}

The paper organization is as follows. We provide a summary of related studies
in Section~\ref{sec:related}.
In Section~\ref{sec:grouping}, we examine the significance of attribute groups to classification accuracy and performance with selected attributes with supervised learning methods.
We then present the new clustering technique incorporating multiple cluster models in Section~\ref{sec:multi_cluster} and evaluation results are presented in Section~\ref{sec:eval}.
Finally we conclude our presentation with a summary and future direction in Section~\ref{sec:conc}.

\section{Related Work and Motivation}
\label{sec:related}
%\subsection{Related Study}
%Existing techniques based on port-numbers are proved to be very ineffective where accuracy is less than 70 \cite{}, as number of applications using random port numbers or non-standard port numbers are increasing day-by-day and also usage of tunneling makes identification of application more difficult just based on port numbers.
%
%To counter drawbacks of port-based application identification, many payload-based techniques have been proposed\cite{}. In payload-based technique, payload of the flow will be extracted and searched for known signature of the applications. Results indicate that this method is very effective with high accuracies. Drawbacks include encrypted traffic, and wide deployment of tools based on payload signature is a problem as many countries doesn't allow the extraction of full payload from packets due to privacy concerns
%
%Drawbacks of both payload-based signature matching and port-based classification led us to use the transport layer characteristics of the application as the differentiator of the application. From various proposed techniques\cite{} we can see combination of transport layer characteristics in combination with Machine learning techniques accurately identifies the applications, with accuracies comparable to that of payload-based signature matching.

A substantial body of research has been conducted for network traffic classification with machine learning techniques.
The work can broadly be divided into the following three categories:

\begin{itemize} \itemsep3pt \parskip3pt \parsep3pt
	\item \emph{Un-supervised}~\cite{}:
	Labeling information of the training data is not available at the time of training. We use various clustering algorithms for the classification of unlabeled data\cite{}.
	\item \emph{Supervised}~\cite{}:
	We provide the labels for the flows when we train the model and then use this model to test each incoming flow whether it belongs to any of the application which is provided at the time of training\cite{}.
	\item \emph{Semi-supervised}~\cite{}: 
	We provide partial labeling information at the time of training and we use clustering algorithms to cluster the training data. We use the partially available labeling information to label each cluster. Heuristics have been proposed on how to label the cluster from partial training information\cite{}
\end{itemize}

%We extensively ran existing techniques\cite{} against our dataset. We ran supervised (Adaboost, Naive Bayes), unsupervised (K-Means using all attributes of the flow statistics) and Early Application Identification (K-Means using only Average Packet Size as the attribute for the machine learning algorithm).

\textbf{ACAS}\cite{ACAS}: In paper\cite{ACAS}, they explored the problem of network traffic classification by \emph{automatically developing signatures} for various network based applications. ACAS encodes initial\textit{n-bytes}of the payload into space and uses supervised machine learning algorithm as the classification algorithm. They did experiments with various settings of the initial \textit{n-bytes} and observed accuracies more than $99\%$ for both settings of the initial bytes. 
We observe this technique is effective in identifying applications with very high accuracy with our data set ($\approx$ 99\%). Drawback of this technique is it requires completely labeled data. Which means we should know prior to the start classification it should know all the applications that classifier may encounter in future (which is not a feasible solution). It also 
As mentioned, however, it requires the access to the payload, which could be limited by laws due to privacy concerns. In addition, encryption plays an important role in classification, which may significantly lower the accuracy.

\textbf{K-Means classification technique}\cite{Erman:2006:TCU:1162678.1162679}. 
%We ran K-Means classification technique, where we use all the available 18 attributes of flow statistics for the classification. We ran experiments to study the effect of the number of clusters\cite{}. 
Main idea of the paper\cite{Erman:2006:TCU:1162678.1162679} is using of K-Means and DBSCAN clustering algorithms which were not used earlier for network traffic classification.  They identified DBSCAN algorithm is the only algorithm which labeled traffic as noise.The time of building the models is very low for K-Means algorithm. They claim K-Means accuracy on an average is around $84\%$.
We ran the K-Means classification technique across by varying the number of clusters from 10 to 140, and Figure~\ref{fig_k_means_all} show the result.
As can be seen from the figure, classification performance is largely not acceptable with quite less than 70\% accuracy.
%In fig.\ref{fig_k_means_all} K-Means all Attrs is the accuracy of the existing technique \cite{} and K-Means Selected is the accuracy when we consider only subset of total attributes, based on the selection strategy which we suggested later in the paper. We observe the accuracy of the existing technique is low in comparison with the accuracy of the clustering technique when we consider only subset of the attributes.


\textbf{Early Application Identification}.\cite{Bernaille:2006:EAI:1368436.1368445}
%We ran Early Application Identification\cite{} against our dataset. Instead of considering \emph{Average Packet Size} for first 4 packets as suggested in the paper\cite{}, we considered whole packets \emph{Average Packet Size} of the flow. It can observed from fig.\ref{fig:early_app_identification} that the accuracy increases slowly with increase in the percent of the training set.
\begin{figure}[!t]
	\centering
	\includegraphics[width=1\columnwidth]{early_application_result}
	% where an .eps filename suffix will be assumed under latex,
	% and a .pdf suffix will be assumed for pdflatex; or what has been declared
	% via \DeclareGraphicsExtensions.
	\caption{\% of training data v/s Accuracy with only average packet size as attribute. We have considered Average Packet Size as the only attribute for K-Means algorithm. We observe that accuracy reaches around 90\% when we have around 95\% of training data}
	\label{early_app_identification}
\end{figure}
In the paper\cite{Bernaille:2006:EAI:1368436.1368445}, they described new technique by which identification of the application can be achieved at the earliest. They discussed the usage of only first \textit{4 packets} in the classification, by which we can identify the application as soon as possible. They propose that by using only Average Packet size of the flow is enough to get desired classification accuracy.They claim the accuracy of 98\% can be achieved by using only Average Packet Size  as the attribute for machine learning algorithms.
We also evaluated this technique. Rather than considering only the first 4 packets in a flow as suggested in the paper, we considered the whole packets to see the maximum performance.
Although this technique is based on clustering as the above technique and uses only the average packet size attribute, it yielded quite enhanced results as shown in Figure~\ref{fig:early_app_identification} across diverse ratios between training data set and testing data set. 

\begin{figure}[!t]
	\centering
	\includegraphics[width=1\columnwidth]{k_means_selected_all}
	% where an .eps filename suffix will be assumed under latex,
	% and a .pdf suffix will be assumed for pdflatex; or what has been declared
	% via \DeclareGraphicsExtensions.
	\caption{Number of clusters v/s Accuracy for K-Means algorithm with All and Selected attributes}
	\label{fig_k_means_all}
\end{figure}

\section{Grouping of Attributes}
\label{sec:grouping}
Total attributes considered for the study are 18. 
\renewcommand{\arraystretch}{1.5}
\begin{table*}
	\caption{Attributes used in out studies}
	\label{table_attributes_description}
	\begin{tabular}{|c|L{2.8in}|L{2.8in}|}
		\hline Name & Description & Related Attributes \\
		\hline IAT  & Inter Arrival Time, time difference between $i^{th}$ and $i+1^{th}$ packets & Standard Deviation IAT\newline Minimum IAT\newline Maximum IAT\newline Average IAT \\
		\hline RIAT & Relative Inter Arrival Time, IAT of the the packet divided my mimimum of IAT & Average RIAT\newline Standard Deviation of RIAT\newline Minimum RIAT\newline Maximum of RIAT \\
		\hline Packet Size & Size of the packet in the flow including headers & Average Packet Size\newline Standard Deviation of Packet Size\newline Minimum Packet Size\newline Maximum Packet Size \\
		\hline Flow Duration & Duration of the entire flow from initiation to termination & Flow Duration\\
		\hline Total Packet Size & Summation of the packet sizes of all the packets inside the flow & Total Packet Size \\
		\hline Number of Packets & Total number of packets inside the flow & Number of Packets \\
		\hline Packets per Second & $ \frac{Number of Packets}{Flow Duration} $ & Average Packets Per Second\\
		\hline Bytes Per Second & $ \frac{Total Packet Size}{Flow Duration} $ & Average Bytes Per Second \\
		\hline Payload Size & $Size_{Paload Size} = Size_{Total} - Size_{header } $ & Payload Size \\
		\hline
	\end{tabular}
	%\begin{tabular}{|c|c|}
	%	\hline  Average IAT & Maximum of IAT \\ [2pt]
	%	\hline  Minimum of IAT & Stadard Deviation of IAT\\ 
	%	\hline  Average RIAT & Maximum of RIAT\\ 
	%	\hline  Minimum of RIAT& Standard Deviation of RIAT\\
	%	\hline  Average Packet Size & Minimum of Packet Size \\  
	%	\hline  Maximum of Packet Size & Standard Deviation of Packet Size \\ 
	%	\hline  Total Packet Size& Flow Duration\\
	%	\hline  Number of Packets & Average Packets/second\\
	%	\hline  Average Bytes/second & Payload Size \\ 
	%	\hline 
%	\end{tabular}
\end{table*}
%\vspace{5pt}
%\begin{itemize} \itemsep3pt \parskip3pt \parsep3pt
%	\item \emph{Flow Size}
%	Total number of bytes transferred during the entire flow.
%	\item \emph{Flow Duration}
%	Difference between time when last packet of the flow captured and first packet of flow captured
%	\item \emph{Packet Size and it's variations}
%	Size of the each packet. We can get \emph{minimum}, \emph{maximum}, \emph{Average} and \emph{Standard deviation} by using all the packets of the flow.
%	\item \emph{Inter Arrival Time and it's variations} 
%	Inter Arrival Time($\tau'$)\cite{} is the time difference between $\tau_{next}$ - $\tau_{current}$.
%	\item \emph{Relative Inter Arrival Time and it's variations} \cite{}
%	Relative inter arrival time is defined as
%	$$\tau'_i = \frac{\tau_i}{\displaystyle \min_{k=1..|f|-1}{\tau_k}}$$
%	Here, $f$ is a flow with $|f|$ number of packets and $\tau_i$ is the time difference between $i$-th packet and $(i+1)$-th packet in that flow.\cite{} 
%\end{itemize}
\subsection{Preliminary Results}
Our experiments results for each group of attributes are as following:
\renewcommand{\arraystretch}{1.5}
\begin{table}
	\caption{Accuracy based on IAT group}
	\label{table_iat_group}
	\begin{tabular}{|c|c|c|c|c|}
		\hline Avg IAT & Min IAT & Max IAT & Std Div IAT & Final Accuracy \\
		\hline \checkmark & $\times$ & $\times$ & $\times$ & 47.29\% \\
		\hline $\times$ & \checkmark & $\times$ & $\times$ & 49.34\% \\
		\hline $\times$ & $\times$ & \checkmark & $\times$ & 39.20\% \\
		\hline $\times$ & $\times$ & $\times$ & \checkmark & 42.82\% \\
		\hline \checkmark & \checkmark & $\times$ & $\times$ & 45.86\% \\
		\hline \checkmark & $\times$ & \checkmark & $\times$ & 44.25\% \\
		\hline \checkmark & $\times$ & $\times$ & \checkmark & 47.78\% \\
		\hline $\times$ & \checkmark & \checkmark & $\times$ & 39.64\% \\
		\hline $\times$ & \checkmark & $\times$ & \checkmark & 42.91\% \\
		\hline $\times$ & $\times$ & \checkmark & \checkmark & 40.54\% \\
		\hline \checkmark & \checkmark & \checkmark & $\times$ & 45.20\% \\
		\hline \checkmark & \checkmark & $\times$ & \checkmark & 47.18\% \\
		\hline \checkmark & $\times$ & \checkmark & \checkmark & 42.70\% \\
		\hline $\times$ & \checkmark & \checkmark & \checkmark & 40.10\% \\
		\hline \checkmark & \checkmark & \checkmark & \checkmark & 42.75\% \\
		\hline 
	\end{tabular}
\end{table}

\renewcommand{\arraystretch}{1.5}
\begin{table}
	\caption{Accuracy based on RIAT group}
	\label{table_riat_group}
	\begin{tabular}{|c|c|c|c|c|}
		\hline Avg RIAT & Max RIAT& Std Div RIAT & Final Accuracy \\
		\hline \checkmark & $\times$ & $\times$ & 55.86\% \\
		\hline $\times$ & \checkmark & $\times$ & 54.08\% \\
		\hline $\times$ & $\times$ & \checkmark & 53.41\% \\
		\hline \checkmark & \checkmark & $\times$ & 55.21\% \\
		\hline \checkmark & $\times$ & \checkmark & 54.28\% \\
		\hline $\times$ & \checkmark & \checkmark & 55.25\% \\
		\hline \checkmark & \checkmark & \checkmark & 56.41\% \\
		\hline 
	\end{tabular}
\end{table}

\renewcommand{\arraystretch}{1.5}
\begin{table}
	\caption{Accuracy based on Packet Size group}
	\label{table_pkt_size_group}
	\tabcolsep=0.01cm
	\begin{tabular}{|c|c|c|c|c|}
		\hline Avg Pkt Size & Min Pkt Size & Max Pkt Size & Std Div Pkt Size & Final Accuracy \\
		\hline \checkmark & $\times$ & $\times$ & $\times$ & 92.43\% \\
		\hline $\times$ & \checkmark & $\times$ & $\times$ & 75.95\% \\
		\hline $\times$ & $\times$ & \checkmark & $\times$ & 92.22\% \\
		\hline $\times$ & $\times$ & $\times$ & \checkmark & 91.82\% \\
		\hline \checkmark & \checkmark & $\times$ & $\times$ & 91.46\% \\
		\hline \checkmark & $\times$ & \checkmark & $\times$ & 92.98\% \\
		\hline \checkmark & $\times$ & $\times$ & \checkmark & 93.96\% \\
		\hline $\times$ & \checkmark & \checkmark & $\times$ & 91.46\% \\
		\hline $\times$ & \checkmark & $\times$ & \checkmark & 92.69\% \\
		\hline $\times$ & $\times$ & \checkmark & \checkmark & 92.76\% \\
		\hline \checkmark & \checkmark & \checkmark & $\times$ & 93.98\% \\
		\hline \checkmark & \checkmark & $\times$ & \checkmark & 93.62\% \\
		\hline \checkmark & $\times$ & \checkmark & \checkmark & 92.76\% \\
		\hline $\times$ & \checkmark & \checkmark & \checkmark & 91.73\% \\
		\hline \checkmark & \checkmark & \checkmark & \checkmark & 93.75\% \\
		\hline 
	\end{tabular}
\end{table}

\renewcommand{\arraystretch}{1.0}
\begin{table*}
	\caption{Accuracy based on Flow attribute group}
	\label{table_flow_group}
	\tabcolsep=0.40cm
	\begin{tabular}{|c|c|C{0.8in}|c|c|c|c|}
		\hline Total Pkt Size & Flow Duration & Number of Packets &  Avg Pkts Per Sec & Avg Bytes Per Sec & Payload Size & Accuracy \\
		%\hline\checkmark & $\times$ & $\times$ & $\times$ & $\times$ & $\times$ & 89.05\% \\
		%\hline$\times$ & \checkmark & $\times$ & $\times$ & $\times$ & $\times$ & 40.72\% \\
		%\hline$\times$ & $\times$ & \checkmark & $\times$ & $\times$ & $\times$ & 59.54\% \\
		%\hline$\times$ & $\times$ & $\times$ & \checkmark & $\times$ & $\times$ & 44.98\% \\
		%\hline$\times$ & $\times$ & $\times$ & $\times$ & \checkmark & $\times$ & 50.48\% \\
	%	\hline$\times$ & $\times$ & $\times$ & $\times$ & $\times$ & \checkmark & 89.43\% \\
		\hline\checkmark & \checkmark & $\times$ & $\times$ & $\times$ & $\times$ & 92.92\% \\
		\hline\checkmark & $\times$ & \checkmark & $\times$ & $\times$ & $\times$ & 91.66\% \\
		\hline\checkmark & $\times$ & $\times$ & \checkmark & $\times$ & $\times$ & 91.12\% \\
		%\hline\checkmark & $\times$ & $\times$ & $\times$ & \checkmark & $\times$ & 49.91\% \\
		\hline\checkmark & $\times$ & $\times$ & $\times$ & $\times$ & \checkmark & 94.53\% \\
		%\hline$\times$ & \checkmark & \checkmark & $\times$ & $\times$ & $\times$ & 60.99\% \\
		%\hline$\times$ & \checkmark & $\times$ & \checkmark & $\times$ & $\times$ & 45.13\% \\
		%\hline$\times$ & \checkmark & $\times$ & $\times$ & \checkmark & $\times$ & 49.29\% \\
		\hline$\times$ & \checkmark & $\times$ & $\times$ & $\times$ & \checkmark & 92.58\% \\
		%\hline$\times$ & $\times$ & \checkmark & \checkmark & $\times$ & $\times$ & 50.00\% \\
		%\hline$\times$ & $\times$ & \checkmark & $\times$ & \checkmark & $\times$ & 50.68\% \\
		\hline$\times$ & $\times$ & \checkmark & $\times$ & $\times$ & \checkmark & 93.80\% \\
		%\hline$\times$ & $\times$ & $\times$ & \checkmark & \checkmark & $\times$ & 49.74\% \\
	%	\hline$\times$ & $\times$ & $\times$ & \checkmark & $\times$ & \checkmark & 87.07\% \\
		%\hline$\times$ & $\times$ & $\times$ & $\times$ & \checkmark & \checkmark & 50.94\% \\
	%	\hline\checkmark & \checkmark & \checkmark & $\times$ & $\times$ & $\times$ & 89.95\% \\
		\hline\checkmark & \checkmark & $\times$ & \checkmark & $\times$ & $\times$ & 92.31\% \\
		%\hline\checkmark & \checkmark & $\times$ & $\times$ & \checkmark & $\times$ & 49.88\% \\
	%	\hline\checkmark & \checkmark & $\times$ & $\times$ & $\times$ & \checkmark & 89.54\% \\
	%	\hline\checkmark & $\times$ & \checkmark & \checkmark & $\times$ & $\times$ & 86.20\% \\
		%\hline\checkmark & $\times$ & \checkmark & $\times$ & \checkmark & $\times$ & 49.81\% \\
	%	\hline\checkmark & $\times$ & \checkmark & $\times$ & $\times$ & \checkmark & 87.25\% \\
		%\hline\checkmark & $\times$ & $\times$ & \checkmark & \checkmark & $\times$ & 50.58\% \\
	%	\hline\checkmark & $\times$ & $\times$ & \checkmark & $\times$ & \checkmark & 88.83\% \\
		%\hline\checkmark & $\times$ & $\times$ & $\times$ & \checkmark & \checkmark & 49.14\% \\
		%\hline$\times$ & \checkmark & \checkmark & \checkmark & $\times$ & $\times$ & 47.74\% \\
		%\hline$\times$ & \checkmark & \checkmark & $\times$ & \checkmark & $\times$ & 49.67\% \\
		\hline$\times$ & \checkmark & \checkmark & $\times$ & $\times$ & \checkmark & 92.02\% \\
		%\hline$\times$ & \checkmark & $\times$ & \checkmark & \checkmark & $\times$ & 49.40\% \\
	%	\hline$\times$ & \checkmark & $\times$ & \checkmark & $\times$ & \checkmark & 89.36\% \\
		%\hline$\times$ & \checkmark & $\times$ & \checkmark & $\times$ & \checkmark & 50.17\% \\
		%\hline$\times$ & $\times$ & \checkmark & \checkmark & \checkmark & $\times$ & 49.45\% \\
	%	\hline$\times$ & $\times$ & \checkmark & \checkmark & $\times$ & \checkmark & 88.46\% \\
		%\hline$\times$ & $\times$ & \checkmark & $\times$ & \checkmark & \checkmark & 51.83\% \\
		%\hline$\times$ & $\times$ & $\times$ & \checkmark & \checkmark & \checkmark & 50.90\% \\
		\hline\checkmark & \checkmark & \checkmark & \checkmark & $\times$ & $\times$ & 91.86\% \\
		%\hline\checkmark & \checkmark & \checkmark & $\times$ & \checkmark & $\times$ & 49.30\% \\
	%	\hline\checkmark & \checkmark & \checkmark & $\times$ & $\times$ & \checkmark & 81.47\% \\
		%\hline\checkmark & \checkmark & $\times$ & \checkmark & \checkmark & $\times$ & 49.64\% \\
	%	\hline\checkmark & \checkmark & $\times$ & \checkmark & $\times$ & \checkmark & 85.54\% \\
	%	\hline\checkmark & \checkmark & $\times$ & $\times$ & \checkmark & \checkmark & 51.12\% \\
	%	\hline\checkmark & $\times$ & \checkmark & \checkmark & \checkmark & $\times$ & 49.76\% \\
	%	\hline\checkmark & $\times$ & \checkmark & \checkmark & $\times$ & \checkmark & 83.67\% \\
	%	\hline\checkmark & $\times$ & \checkmark & $\times$ & \checkmark & \checkmark & 50.40\% \\
	%	\hline\checkmark & $\times$ & $\times$ & \checkmark & \checkmark & \checkmark & 50.46\% \\
	%	\hline$\times$ & \checkmark & \checkmark & \checkmark & \checkmark & $\times$ & 50.33\% \\
	%	\hline$\times$ & \checkmark & \checkmark & \checkmark & $\times$ & \checkmark & 87.11\% \\
	%	\hline$\times$ & \checkmark & \checkmark & $\times$ & \checkmark & \checkmark & 51.16\% \\
	%	\hline$\times$ & \checkmark & $\times$ & \checkmark & \checkmark & \checkmark & 51.54\% \\
	%	\hline$\times$ & $\times$ & \checkmark & \checkmark & \checkmark & \checkmark & 51.18\% \\
	%	\hline\checkmark & \checkmark & \checkmark & \checkmark & \checkmark & $\times$ & 49.20\% \\
	%	\hline\checkmark & \checkmark & \checkmark & \checkmark & $\times$ & \checkmark & 80.33\% \\
	%	\hline\checkmark & \checkmark & \checkmark & $\times$ & \checkmark & \checkmark & 49.41\% \\
	%	\hline\checkmark & \checkmark & $\times$ & \checkmark & \checkmark & \checkmark & 50.08\% \\
	%	\hline\checkmark & $\times$ & \checkmark & \checkmark & \checkmark & \checkmark & 51.12\% \\
	%	\hline$\times$ & \checkmark & \checkmark & \checkmark & \checkmark & \checkmark & 51.92\% \\
	%	\hline\checkmark & \checkmark & \checkmark & \checkmark & \checkmark & \checkmark & 50.93\% \\
		\hline
	\end{tabular}
\end{table*}
We observe from Table.\ref{table_iat_group}, Table.\ref{table_riat_group}, Table.\ref{table_flow_group} and Table.\ref{table_pkt_size_group} that accuracy is not very good when we consider all the attributes of flow in the classification. When subset of attributes in a group is considered then the accuracy is better than when all attributes in the group are considered. From Table.\ref{table_iat_group}, Table.\ref{table_riat_group} accuracy is very low, this group doesn't make much difference in the accuracy when considered in the classification. From Table.\ref{table_flow_group} and Table.\ref{table_pkt_size_group} we observe accuracy when considering all the attributes is not as good as when we consider subset of them. Table.\ref{table_pkt_size_group} we observe combination of \emph{Avg Pkt Size} and \emph{Std Div Pkt Size} gives the best accuracy of the group, similarly combination of \emph{Min Pkt Size}, \emph{Max Pkt Size} and \emph{Avg Pkt Size} gives second best accuracy of the group. Similarly from Table.\ref{table_flow_group} combination of \emph{Total Pkt Size}, \emph{Payload Size} and also combination of \emph{Payload size}, \emph{Number of Packets in the flow} gives better accuracy in the group. 
\begin{figure}[!t]
	\centering
	\includegraphics[width=1\columnwidth]{selected_total_attr_adaboost}
	% where an .eps filename suffix will be assumed under latex,
	% and a .pdf suffix will be assumed for pdflatex; or what has been declared
	% via \DeclareGraphicsExtensions.
	\caption{Percentage of Training data v/s Accuracy for Adaboost}
	\label{adaboost_selected_total}
\end{figure}
From fig.\ref{adaboost_selected_total}, we observe with only subset (7 out 18 attributes) is giving almost same accuracy(at some points even more), as when we consider total 18 attributes with Adaboost (Supervised) Algorithm\cite{}. Extra attributes is not increasing accuracy at all. With less attributes we have faster classifier. From fig.\ref{fig_k_means_all} we observe selection of subset of attributes increases accuracy dramatically. 

\section{Multiple Cluster Models}
\label{sec:multi_cluster}
\subsection{Population Fraction}
In this paper, we introduced new term called population fraction, which stands for percentage of dominant application (application with most number of flows in the cluster in consideration.) flows to total flows in a cluster
\begin{equation}
P_{clus} = (\frac{flows_{dominant}}{flows_{total}})_{clus}
\end{equation}
\subsection{Description of the technique used for classification}
Based on \ref{table_flow_group} and \ref{table_pkt_size_group} we can observe particular combination of attributes results in better accuracy(like \emph{Average Pkt Size} and \emph{Std Div Pkt Size}). From fig.\ref{fig_k_means_all} it can be observed that the accuracy of selected attributes is around 90\%, which can be further improved by considering multiple trained models(with different combination of attributes) than with one trained model(which has superset of combination of attributes used in multiple models). In a single classifier, supervised or unsupervised, we cannot consider combination of attributes as we have to put every attribute to be considered through training model. So, we advised usage of four different models, each with particular combination of attributes. 
\begin{itemize}
	\item \emph{Model 1}
	In this model we considered attributes, Average Pkt Size and Standard Pkt Size.
	\item \emph{Model 2}
	We considered Average, Minimum, Maximum Packet size attributes in this model.
	\item \emph{Model 3}
	Considered Total Flow Size, Payload Size as the attributes for this model
	\item \emph{Model 4}
	Considered Number of Packets, Payload Size as the attributes for this model.
\end{itemize}
We use results from four models in determining the final classified result of the incoming flow. We tested different hypothesis. Following are the explanation of hypothesis considered in deciding the final classification result.
\emph{Considered Strategies}
\begin{itemize} \itemsep3pt \parskip3pt \parsep3pt
	\item \emph{Random}
	Select classification of any of the results from the four models as the final classification result
	\item \emph{Greatest}
	Select classification result from model with highest population fraction as the final classification result
	\item \emph{Quorum}
	Select the majority result from trained models. If we have 3 models resulted in one classification result and other resulted in other application then we consider result of 3 as the final classification result. In case of tie we select application randomly
	\item \emph{Unanimous}
	Select the final classification only when all the results from the four models exactly results in the same result. Otherwise mark it as unknown
	\item \emph{Unanimous Greatest}
	Similar to unanimous but fallback hypothesis is greatest.
	\item \emph{Unanimous Random}
	Similar to unanimous but fallback hypothesis is random
	\item \emph{Unanimous Quorum}
	Unanimous with fallback hypothesis as Quorum
\end{itemize}

\section{Evaluation}
\label{sec:eval}
\subsection{Datasets}
Data had been collected with full payload in early-2014. It has been gathered on various interfaces 1) Wired 2) WIFI 3) 3G and LTE. Data had been collected for individual application in isolation, by generating requests intentionally and capturing bidirectional data.  Considered five tuples (i.e., Source IP, destination IP, source port, destination port and protocol). Used TCP flags to mark the start and end of flow(flow started before the capture, or flows terminating after the capture are not considered in construction of flows). 

We selected 5 protocols (Skype, Bittorrent, Http, Edonkey and Gnutella) for further study, criteria for the selection of this protocols is number of flows. 
\subsection{Cleaning}
We constructed flows from packets, which we read from \emph{pcap} files using \emph{scapy} a python library. Used community maintained signatures\cite{} available for this protocols from L7 filters. We matched payload of each constructed flow with corresponding signatures from L7, if we find an unmatched flow then it is discarded otherwise labeled as the matched signature.

\subsection{Experimental Setup}
We used \emph{sklearn} a python library for machine learning algorithms in our study. \emph{sklearn} is the most used python library for data analysis in python. Used \emph{Numpy} for numerical computations, it is a python library to handle numerical computation in efficient way. Used \emph{Matplotlib} for plotting the results, it is a python library which lets us plot the results. 

We divide our data into training and testing partitions. We used training data to train the model and testing data to test the accuracy of the trained model in correctly classifying protocols. 
During training phase we label clusters in each model(we have 4 of them) based on population fraction(majority based). Each test flow is fed into the trained model and we get the label for that test flow which is compared with the actual label for that flow. Finally 

\begin{equation}
	Accuracy = \frac{TP}{Total Flows}
\end{equation}
 Where $TP$ stands for True positive, meaning correctly identified flow during testing. Total flows are the total number of flows in the testing set.

\subsection{Results}
\begin{figure}[!t]
	\centering
	\includegraphics[width=1\columnwidth]{adaboost_all_attr_plot.png}
	% where an .eps filename suffix will be assumed under latex,
	% and a .pdf suffix will be assumed for pdflatex; or what has been declared
	% via \DeclareGraphicsExtensions.
	\caption{Percent of training data v/s Accuracy for Adaboost with Groups of Attributes.Plotting Adaboost with each group of attributes.}
	\label{fig:adaboost_attributes}
\end{figure}

\begin{figure}[!t]
	\centering
	\includegraphics[width=1\columnwidth]{naive_bayes_all_attr_plot.png}
	% where an .eps filename suffix will be assumed under latex,
	% and a .pdf suffix will be assumed for pdflatex; or what has been declared
	% via \DeclareGraphicsExtensions.
	\caption{\% of training data v/s Accuracy for Naive Bayes with Groups of Attributes. Plotting Naive Bayes with each group of attributes.}
	\label{fig:naive_attributes}
\end{figure}
From fig.\ref{fig:adaboost_attributes} we notice classification with highest accuracy at almost all the ratios of training set is observed for Adaboost. Accuracy for Naive Bayes\cite{} from fig.\ref{fig:naive_attributes} is very low when compared with Adaboost, it is not very effective in classifying the protocols.

\begin{figure}[!t]
	\centering
	\includegraphics[width=1\columnwidth]{hypothesis_updated}
	% where an .eps filename suffix will be assumed under latex,
	% and a .pdf suffix will be assumed for pdflatex; or what has been declared
	% via \DeclareGraphicsExtensions.
	\caption{\# Clusters v/s Accuracy of various strategies }
	\label{fig:hypo_updated}
\end{figure}
From fig.\ref{fig:hypo_updated}, it can noticed that accuracies are very high for \emph{Quorum}, \emph{Greatest} and \emph{Unanimous Greatest} strategies.

We chose \emph{Unanimous Greatest} for further study. Results for proposed classification technique, Supervised(Adaboost) and K-Means are used for comparison.
\begin{figure}[!t]
	\centering
	\includegraphics[width=1\columnwidth]{super_hypo_accuracy_percent_training}
	% where an .eps filename suffix will be assumed under latex,
	% and a .pdf suffix will be assumed for pdflatex; or what has been declared
	% via \DeclareGraphicsExtensions.
	\caption{Percent of training data v/s Accuracy Comparing all the existing and proposed technique}
	\label{fig:super_hypo}
\end{figure}
From fig.\ref{fig:super_hypo} we notice that the accuracy is higher than traditional K-Means clustering algorithm at almost all the ratios of the training set. Proposed technique accuracy is comparable to that of the supervised(Adaboost) technique, at times it is higher than supervised too.

% An example of a floating figure using the graphicx package.
% Note that \label must occur AFTER (or within) \caption.
% For figures, \caption should occur after the \includegraphics.
% Note that IEEEtran v1.7 and later has special internal code that
% is designed to preserve the operation of \label within \caption
% even when the captionsoff option is in effect. However, because
% of issues like this, it may be the safest practice to put all your
% \label just after \caption rather than within \caption{}.
%
% Reminder: the "draftcls" or "draftclsnofoot", not "draft", class
% option should be used if it is desired that the figures are to be
% displayed while in draft mode.
%

% Note that IEEE typically puts floats only at the top, even when this
% results in a large percentage of a column being occupied by floats.


% An example of a double column floating figure using two subfigures.
% (The subfig.sty package must be loaded for this to work.)
% The subfigure \label commands are set within each subfloat command, the
% \label for the overall figure must come after \caption.
% \hfil must be used as a separator to get equal spacing.
% The subfigure.sty package works much the same way, except \subfigure is
% used instead of \subfloat.
%
%\begin{figure*}[!t]
%\centerline{\subfloat[Case I]\includegraphics[width=2.5in]{subfigcase1}%
%\label{fig_first_case}}
%\hfil
%\subfloat[Case II]{\includegraphics[width=2.5in]{subfigcase2}%
%\label{fig_second_case}}}
%\caption{Simulation results}
%\label{fig_sim}
%\end{figure*}
%
% Note that often IEEE papers with subfigures do not employ subfigure
% captions (using the optional argument to \subfloat), but instead will
% reference/describe all of them (a), (b), etc., within the main caption.


% An example of a floating table. Note that, for IEEE style tables, the 
% \caption command should come BEFORE the table. Table text will default to
% \footnotesize as IEEE normally uses this smaller font for tables.
% The \label must come after \caption as always.
%
%\begin{table}[!t]
%% increase table row spacing, adjust to taste
%\renewcommand{\arraystretch}{1.3}
% if using array.sty, it might be a good idea to tweak the value of
% \extrarowheight as needed to properly center the text within the cells
%\caption{An Example of a Table}
%\label{table_example}
%\centering
%% Some packages, such as MDW tools, offer better commands for making tables
%% than the plain LaTeX2e tabular which is used here.
%\begin{tabular}{|c||c|}
%\hline
%One & Two\\
%\hline
%Three & Four\\
%\hline
%\end{tabular}
%\end{table}


% Note that IEEE does not put floats in the very first column - or typically
% anywhere on the first page for that matter. Also, in-text middle ("here")
% positioning is not used. Most IEEE journals/conferences use top floats
% exclusively. Note that, LaTeX2e, unlike IEEE journals/conferences, places
% footnotes above bottom floats. This can be corrected via the \fnbelowfloat
% command of the stfloats package.



\section{Conclusion}
\label{sec:conc}
By considering all the attributes of flow in classifying the application doesn't give better results. Selected attributes with combinations gives better accuracy. Accuracy doesn't continually increases along with number of clusters, as we get better accuracy when we are around 40 clusters. 

Considering IAT and RIAT for classification gives us the least accuracy when compared with other attributes of the flow. Even considering all attributes of packet size and flow also doesn't give better accuracy. 

Population fraction will be very good parameter in classification of flow using clustering techniques. 


% conference papers do not normally have an appendix


% use section* for acknowledgement
\section*{Acknowledgment}
This work was in part supported by the IT R\&D program of MOTIE/KEIT
[10041548,
``240Gbps realtime automatic signature generation
system for application traffic classification supporting over 95\%
completeness and accuracy'']



% trigger a \newpage just before the given reference
% number - used to balance the columns on the last page
% adjust value as needed - may need to be readjusted if
% the document is modified later
%\IEEEtriggeratref{8}
% The "triggered" command can be changed if desired:
%\IEEEtriggercmd{\enlargethispage{-5in}}

% references section

% can use a bibliography generated by BibTeX as a .bbl file
% BibTeX documentation can be easily obtained at:
% http://www.ctan.org/tex-archive/biblio/bibtex/contrib/doc/
% The IEEEtran BibTeX style support page is at:
% http://www.michaelshell.org/tex/ieeetran/bibtex/
\bibliographystyle{IEEEtranS}
% argument is your BibTeX string definitions and bibliography database(s)
\bibliography{IEEEabrv,references}
%
% <OR> manually copy in the resultant .bbl file
% set second argument of \begin to the number of references
% (used to reserve space for the reference number labels box)
%\begin{thebibliography}{1}

%\bibitem{IEEEhowto:kopka}
%H.~Kopka and P.~W. Daly, \emph{A Guide to \LaTeX}, 3rd~ed.\hskip 1em plus
%  0.5em minus 0.4em\relax Harlow, England: Addison-Wesley, 1999.

%\end{thebibliography}




% that's all folks
\end{document}


